\documentclass[11pt]{book}
% document-level settings
% \setcounter{secnumdepth}{-1} % don't even number chapters!
\setcounter{tocdepth}{0} % put sections in table of contents
\renewcommand{\MakeUppercase}[1]{\textsc{#1}} % helps with headings
\special{papersize=8.5in,11in}
\setlength{\pdfpagewidth}{8.5in}
\setlength{\pdfpageheight}{11in}
\setlength{\textwidth}{4in}
\setlength{\textheight}{6in}
\setlength{\oddsidemargin}{3.25in}
\addtolength{\oddsidemargin}{-0.5\textwidth}
\setlength{\evensidemargin}{\oddsidemargin}

\sloppy

% chapter settings
\newcommand{\chapternames}{\chaptername s}

\includeonly{}
\begin{document}

\chapter{Tear down the school}

Depending on how you count it, I have spent between ten and twenty
years as a tertiary educator, mainly teaching introductory physics
courses to American students of excellent to mediocre calibre.  If
there is one thing I have learned, it is that education at all levels
in America (and, indeed, probably most of the World) is a failure.

This failure is systemic.  No incremental changes can fix it.  In what
follows, I will describe this failure---it is so multi-dimensional it
requires an entire book to express---but I will do nothing else.  I
will not explain it and I will not propose any useful way to fix it.
I doubt, in fact, that it \emph{can} be fixed, in the context of
contemporary American society, where the idea of school and its
objectives, its architecture, its place in society, its financial
support, and its limitations are fixedly held, even by its most
allegedly radical reformers.  Every student knows why he or she is in
school, every teacher knows what he or she is doing, every parent
knows what he or she is getting, every taxpayer knows what he or she
is supporting.  That knowledge supports the schools, to be sure, but
it also freezes them in a carbonite trap, in which the only possible
changes we can make are those that make school less likely to bring
real education to students or real new opportunities to our society.
Or, more precisely, educational reforms that incrementally improve one
aspect of education have so many negative consequences in every other
aspect that there is no room for celebration.

After that introduction, a reader might conclude that I am against
research and reform aimed at improving education.  I am not.  It is
not the fault of educators, researchers, and reformers that school is
situated in a political, economic, and aesthetic culture that limits
completely their achievements.  And, more importantly, I am not
railing against \emph{education}.  On the contrary: Would I write a
manifesto against schools if I did not care deeply about education?
Every project aimed at understanding or improving schools brings new
ideas and new information about how students learn, how schools fail,
and what forces outside of the classroom support the barriers to true
education.

You can't claim that schools fail if you don't have a clear picture of
what success might be.  That is, in order to assess school, we must
agree on the \emph{objectives}.  The objectives I set for school are
not controversial.  Of course they won't be exactly the objectives of
those who attend, teach, run, or fund schools, but they will be close.
I give them here, in order more-or-less of descending importance (to
me).

I believe that school should develop---more than knowledge itself---a
love of understanding, learning, and knowledge, and the skills a
student needs to pursue the objects of that love.  If school produced
a generation of students who loved knowledge and knew how to obtain
it, it would produce a generation with intellectual curiosity and a
desire to understand the world and its problems.  It would produce a
generation of lifetime learners.  In \chaptername~\ref{chap:hate}, I
will discuss the myriad ways in which schools fail disastrously to
meet this objective.  One famous example of this is the large fraction
of students---and larger fraction of girls---who emerge from high
school hating and fearing mathematics.  Which is worse, never having
been taught math in school and discovering its beauty only later in
life, or being taught it for twelve years and then hating and avoiding
it for all the rest?

I believe that school should act to equalize opportunity in a society
of great disparity of opportunity.  Students have, by the conditions
of their birth---for which, in my view, they ought not be punished or
rewarded---very different opportunities or likelihoods for various
kinds of success, in learning and in life.  Of course, not all of the
causes of this are understood, and school cannot be expected to right
all the wrongs of society.  However, it should go some way towards
giving opportunities to those who wouldn't have them otherwise.  In
\chaptername~\ref{chap:disparity}, I will remind my readers that in
this regard also, school is an almost complete failure.  In some ways,
this is the most American of the problems I am discussing, as not all
western societies have reserved their worst schools for the people who
least deserve them.  I put this extremely important objective after
the love-of-learning objective only for the reason that I (a utopian)
believe that if we achieved the former, we would unavoidably make
progress on this one.

I believe that children, in school, should learn to interact richly
and kindly with their fellows.  They should learn how to ask for what
they want to get, give what they are comfortable giving, protect
themselves and those weaker from aggressors, and use their strengths
for good.  These interpersonal capabilities are learned throughout
life and once again we can't ask school to do it all, but I think we
can ask school to contribute---positively.  Certainly most students
\emph{do} learn many of these things during their school years, but I
will argue in \chaptername~\ref{chap:social} that it is despite
school, not because of it.  In school, children are not permitted to
speak to one another for most of the day, and yet its proponents hail
it as a socializer.  Indeed, school cannot help but socialize
children: It has a monopoly on children's time and attention.  It is
the positive influence on this socialization that is difficult to
locate.  Negative influences are easy to find.

I believe that the tasks and activities that students perform in
school should be directly and immediately valuable to the students, at
least in their majority.  The activities performed by the students
should be interesting to them, engaging, fun, and have a meaning for
them in the present.  In \chaptername~\ref{chap:why}, I will point out
the obvious and obviously disturbing point that the justification for
almost everything that is taught in school is its importance for
\emph{future schooling}.  School has been carefully constructed as a
kind of purgatory in which students learn decontextualized material
whose purpose and meaning is only made clear to those who excel to the
end of all schooling (a university degree or more) and, remarkably,
remember enough of their past to connect the dots.  A more direct way
of pointing out the failure of school on this point is to note that
there is almost no assignment that, by the canceling of it, a teacher
wouldn't make most of her or his class extremely happy.  Yet another
is to point out that it is common to explain to students doing poorly
in school that their behavior is reducing their chances for college;
when will we realize that it is absurd to justify working hard at
unpleasant tasks in school by promising more of the same in reward?

I believe that, in school, students should learn not just how things
\emph{are} but how to \emph{make them so}.  Students ought to emerge
from school as doers who are not afraid to build things, experiment,
get dirty, embark on long projects, or create novel objects, systems,
and organizations.  If school produced a generation of doers, it would
produce a generation of entrepreneurs, engineers, scientists,
organizers, and activists, who could simultaneously make our society
better off and just plain better.  In \chaptername~\ref{chap:book}, I
will assess schools on this objective.  Guess what?  They fail on this
too.  Indeed, by the end of a dozen years of school, students believe
that facts live in in books and that their independent verification or
usefulness is outside their realm entirely.  Every high-school student
knows that a water molecule is made up of two hydrogen atoms and one
oxygen atom, but almost none could independently make a plan (let
alone execute it) to establish that ``fact'' by her or himself.
High-school chemistry establishes a set of rules for chemistry land;
the real world works by chaotic magic.

So school fails to meet my objectives, so what?  I don't have any
answer to that.  Make different objectives and decide that it
succeeds!  I don't actually blame the schools themselves.  Schools are
run by and classes are taught by some of the most creative,
thoughtful, and devoted people in our society.  I can't think of a
better group of people to be taking care of our children.  I have
rarely encountered a teacher I thought was doing a bad job.  The
failure is not a failure of teachers or principals.  It is a system
failure.  We have a certain idea of school in our collective
unconscious head and that idea is very wrong.  But it is reinforced by
every facet of society, every movie, every conversation, every
admissions event, every back-to-school sale, every job interview, and
every biography or memoir.  There is no reforming school by
incremental change.  If we want to change school we have to change who
we \emph{are}.  That change starts with conversations.  This book is
an attempt to start those conversations.

\chapter{Learn to hate}\label{chap:hate}

\chapter{The great de-leveler}\label{chap:disparity}

\chapter{Anti-socialization}\label{chap:social}

\chapter{Why go to school?}\label{chap:why}

\chapter{Book learnin'}\label{chap:book}

\chapter{Reform is futile}


\end{document}

\documentclass[12pt]{article}
\newcommand{\tablesize}{\footnotesize\sffamily}
\begin{document}

\subsection*{The hundreds chart}

\noindent
\textsl{by David W. Hogg, New York City, 2011 June 7}
\vspace{1ex}

V wasn't showing a lot of interest in math, but she was showing a lot
of interest in ``playing school'' at home.  J had the brilliant
insight that all V needed was a hundreds chart like her teacher
Emily's.  It worked like a charm; V became much more interested in
math overnight and does math problems and notices patterns in the
chart voluntarily and without any prompting from us.  But \emph{I}
also love the chart.  Here I set down a few little thoughts about it.

Here's the way the usual hundreds chart is arranged in most
classrooms:
\begin{center}\tablesize
\begin{tabular}{|c|c|c|c|c|c|c|c|c|c|}
\hline
  1 &  2 &  3 &  4 &  5 &  6 &  7 &  8 &  9 & 10 \\ \hline
 11 & 12 & 13 & 14 & 15 & 16 & 17 & 18 & 19 & 20 \\ \hline
 21 & 22 & 23 & 24 & 25 & 26 & 27 & 28 & 29 & 30 \\ \hline
 31 & 32 & 33 & 34 & 35 & 36 & 37 & 38 & 39 & 40 \\ \hline
 41 & 42 & 43 & 44 & 45 & 46 & 47 & 48 & 49 & 50 \\ \hline
 51 & 52 & 53 & 54 & 55 & 56 & 57 & 58 & 59 & 60 \\ \hline
 61 & 62 & 63 & 64 & 65 & 66 & 67 & 68 & 69 & 70 \\ \hline
 71 & 72 & 73 & 74 & 75 & 76 & 77 & 78 & 79 & 80 \\ \hline
 81 & 82 & 83 & 84 & 85 & 86 & 87 & 88 & 89 & 90 \\ \hline
 91 & 92 & 93 & 94 & 95 & 96 & 97 & 98 & 99 & 100 \\ \hline
\end{tabular}
\end{center}

Imagine instead that the chart started not in the top-left corner,
but one space to the right of that:
\begin{center}\tablesize
\begin{tabular}{|c|c|c|c|c|c|c|c|c|c|}
\hline
    &  1 &  2 &  3 &  4 &  5 &  6 &  7 &  8 &  9 \\ \hline
 10 & 11 & 12 & 13 & 14 & 15 & 16 & 17 & 18 & 19 \\ \hline
 20 & 21 & 22 & 23 & 24 & 25 & 26 & 27 & 28 & 29 \\ \hline
 30 & 31 & 32 & 33 & 34 & 35 & 36 & 37 & 38 & 39 \\ \hline
 40 & 41 & 42 & 43 & 44 & 45 & 46 & 47 & 48 & 49 \\ \hline
 50 & 51 & 52 & 53 & 54 & 55 & 56 & 57 & 58 & 59 \\ \hline
 60 & 61 & 62 & 63 & 64 & 65 & 66 & 67 & 68 & 69 \\ \hline
 70 & 71 & 72 & 73 & 74 & 75 & 76 & 77 & 78 & 79 \\ \hline
 80 & 81 & 82 & 83 & 84 & 85 & 86 & 87 & 88 & 89 \\ \hline
 90 & 91 & 92 & 93 & 94 & 95 & 96 & 97 & 98 & 99 \\ \hline
\end{tabular}
\end{center}
With this small change, the tens column (10, 20, 30, and so on) is on
the left instead of the right.  To my eye, it looks nicer, because,
for example, on the third row, every number starts with the digit
``2''.  Previously, they all began with ``2'' except the last number
(30), which begins with ``3''.

I love that every number on each row begins with the same digit, and
every number in each column ends with the same digit.  It is like the
second digit gives the ``$x$'' (horizontal) position, and the first
gives the ``$y$'' (vertical) position, as in a Cartesian coordinate
system.  This symmetry suggests typesetting the first row with
zero-prefaced numbers, like this:
\begin{center}\tablesize
\begin{tabular}{|c|c|c|c|c|c|c|c|c|c|}
\hline
    & 01 & 02 & 03 & 04 & 05 & 06 & 07 & 08 & 09 \\ \hline
 10 & 11 & 12 & 13 & 14 & 15 & 16 & 17 & 18 & 19 \\ \hline
 20 & 21 & 22 & 23 & 24 & 25 & 26 & 27 & 28 & 29 \\ \hline
 30 & 31 & 32 & 33 & 34 & 35 & 36 & 37 & 38 & 39 \\ \hline
 40 & 41 & 42 & 43 & 44 & 45 & 46 & 47 & 48 & 49 \\ \hline
 50 & 51 & 52 & 53 & 54 & 55 & 56 & 57 & 58 & 59 \\ \hline
 60 & 61 & 62 & 63 & 64 & 65 & 66 & 67 & 68 & 69 \\ \hline
 70 & 71 & 72 & 73 & 74 & 75 & 76 & 77 & 78 & 79 \\ \hline
 80 & 81 & 82 & 83 & 84 & 85 & 86 & 87 & 88 & 89 \\ \hline
 90 & 91 & 92 & 93 & 94 & 95 & 96 & 97 & 98 & 99 \\ \hline
\end{tabular}
\end{center}
This also suggests thinking about the ``transpose'' chart:
\begin{center}\tablesize
\begin{tabular}{|c|c|c|c|c|c|c|c|c|c|}
\hline
    & 10 & 20 & 30 & 40 & 50 & 60 & 70 & 80 & 90 \\ \hline
 01 & 11 & 21 & 31 & 41 & 51 & 61 & 71 & 81 & 91 \\ \hline
 02 & 12 & 22 & 32 & 42 & 52 & 62 & 72 & 82 & 92 \\ \hline
 03 & 13 & 23 & 33 & 43 & 53 & 63 & 73 & 83 & 93 \\ \hline
 04 & 14 & 24 & 34 & 44 & 54 & 64 & 74 & 84 & 94 \\ \hline
 05 & 15 & 25 & 35 & 45 & 55 & 65 & 75 & 85 & 95 \\ \hline
 06 & 16 & 26 & 36 & 46 & 56 & 66 & 76 & 86 & 96 \\ \hline
 07 & 17 & 27 & 37 & 47 & 57 & 67 & 77 & 87 & 97 \\ \hline
 08 & 18 & 28 & 38 & 48 & 58 & 68 & 78 & 88 & 98 \\ \hline
 09 & 19 & 29 & 39 & 49 & 59 & 69 & 79 & 89 & 99 \\ \hline
\end{tabular}
\end{center}
Nice!

Zero-prefacing the top row (or left column in the transpose chart)
makes sense from a symmetry sense, and also permits discovery of the
idea that a one-digit number has zero in the tens position implicitly.
It also suggests filling in the top-left blank with the only possible
number: It has to have a zero in the first position, and a zero in the
second position:
\begin{center}\tablesize
\begin{tabular}{|c|c|c|c|c|c|c|c|c|c|}
\hline
 00 & 01 & 02 & 03 & 04 & 05 & 06 & 07 & 08 & 09 \\ \hline
 10 & 11 & 12 & 13 & 14 & 15 & 16 & 17 & 18 & 19 \\ \hline
 20 & 21 & 22 & 23 & 24 & 25 & 26 & 27 & 28 & 29 \\ \hline
 30 & 31 & 32 & 33 & 34 & 35 & 36 & 37 & 38 & 39 \\ \hline
 40 & 41 & 42 & 43 & 44 & 45 & 46 & 47 & 48 & 49 \\ \hline
 50 & 51 & 52 & 53 & 54 & 55 & 56 & 57 & 58 & 59 \\ \hline
 60 & 61 & 62 & 63 & 64 & 65 & 66 & 67 & 68 & 69 \\ \hline
 70 & 71 & 72 & 73 & 74 & 75 & 76 & 77 & 78 & 79 \\ \hline
 80 & 81 & 82 & 83 & 84 & 85 & 86 & 87 & 88 & 89 \\ \hline
 90 & 91 & 92 & 93 & 94 & 95 & 96 & 97 & 98 & 99 \\ \hline
\end{tabular}
\end{center}

But one thing I like about this hundreds chart is that on the
hundredth day of school, you don't complete this chart (that happened
on the 99th day); you \emph{start over}, by prepending ``1'' digits in
front of numbers.  This is what it could look like on the 100th day:
\begin{center}\tablesize
\begin{tabular}{|c|c|c|c|c|c|c|c|c|c|}
\hline
100 & 01 & 02 & 03 & 04 & 05 & 06 & 07 & 08 & 09 \\ \hline
 10 & 11 & 12 & 13 & 14 & 15 & 16 & 17 & 18 & 19 \\ \hline
 20 & 21 & 22 & 23 & 24 & 25 & 26 & 27 & 28 & 29 \\ \hline
 30 & 31 & 32 & 33 & 34 & 35 & 36 & 37 & 38 & 39 \\ \hline
 40 & 41 & 42 & 43 & 44 & 45 & 46 & 47 & 48 & 49 \\ \hline
 50 & 51 & 52 & 53 & 54 & 55 & 56 & 57 & 58 & 59 \\ \hline
 60 & 61 & 62 & 63 & 64 & 65 & 66 & 67 & 68 & 69 \\ \hline
 70 & 71 & 72 & 73 & 74 & 75 & 76 & 77 & 78 & 79 \\ \hline
 80 & 81 & 82 & 83 & 84 & 85 & 86 & 87 & 88 & 89 \\ \hline
 90 & 91 & 92 & 93 & 94 & 95 & 96 & 97 & 98 & 99 \\ \hline
\end{tabular}
\end{center}
On the 114th day, it could look like this:
\begin{center}\tablesize
\begin{tabular}{|c|c|c|c|c|c|c|c|c|c|}
\hline
100 &101 &102 &103 &104 &105 &106 &107 &108 &109 \\ \hline
110 &111 &112 &113 &114 & 15 & 16 & 17 & 18 & 19 \\ \hline
 20 & 21 & 22 & 23 & 24 & 25 & 26 & 27 & 28 & 29 \\ \hline
 30 & 31 & 32 & 33 & 34 & 35 & 36 & 37 & 38 & 39 \\ \hline
 40 & 41 & 42 & 43 & 44 & 45 & 46 & 47 & 48 & 49 \\ \hline
 50 & 51 & 52 & 53 & 54 & 55 & 56 & 57 & 58 & 59 \\ \hline
 60 & 61 & 62 & 63 & 64 & 65 & 66 & 67 & 68 & 69 \\ \hline
 70 & 71 & 72 & 73 & 74 & 75 & 76 & 77 & 78 & 79 \\ \hline
 80 & 81 & 82 & 83 & 84 & 85 & 86 & 87 & 88 & 89 \\ \hline
 90 & 91 & 92 & 93 & 94 & 95 & 96 & 97 & 98 & 99 \\ \hline
\end{tabular}
\end{center}

Going back to the basic hundreds chart, to wit:
\begin{center}\tablesize
\begin{tabular}{|c|c|c|c|c|c|c|c|c|c|}
\hline
 00 & 01 & 02 & 03 & 04 & 05 & 06 & 07 & 08 & 09 \\ \hline
 10 & 11 & 12 & 13 & 14 & 15 & 16 & 17 & 18 & 19 \\ \hline
 20 & 21 & 22 & 23 & 24 & 25 & 26 & 27 & 28 & 29 \\ \hline
 30 & 31 & 32 & 33 & 34 & 35 & 36 & 37 & 38 & 39 \\ \hline
 40 & 41 & 42 & 43 & 44 & 45 & 46 & 47 & 48 & 49 \\ \hline
 50 & 51 & 52 & 53 & 54 & 55 & 56 & 57 & 58 & 59 \\ \hline
 60 & 61 & 62 & 63 & 64 & 65 & 66 & 67 & 68 & 69 \\ \hline
 70 & 71 & 72 & 73 & 74 & 75 & 76 & 77 & 78 & 79 \\ \hline
 80 & 81 & 82 & 83 & 84 & 85 & 86 & 87 & 88 & 89 \\ \hline
 90 & 91 & 92 & 93 & 94 & 95 & 96 & 97 & 98 & 99 \\ \hline
\end{tabular}
\end{center}
as a computer scientist (in part) I can't help notice that the
top-left $2\times 2$, $3\times 3$, and $4\times 4$ blocks of the
chart look like this:
\begin{center}\tablesize
\begin{tabular}{|c|c|}
\hline
 00 & 01 \\ \hline
 10 & 11 \\ \hline
\end{tabular}
\quad
\begin{tabular}{|c|c|c|}
\hline
 00 & 01 & 02 \\ \hline
 10 & 11 & 12 \\ \hline
 20 & 21 & 22 \\ \hline
\end{tabular}
\quad
\begin{tabular}{|c|c|c|c|}
\hline
 00 & 01 & 02 & 03 \\ \hline
 10 & 11 & 12 & 13 \\ \hline
 20 & 21 & 22 & 23 \\ \hline
 30 & 31 & 32 & 33 \\ \hline
\end{tabular}
\end{center}
These are the binary, trinary, and base-4 ``hundreds'' charts, or, if
you prefer, the binary fours chart, trinary nines chart, and base-4
sixteens chart.  And so on.  Also, the decimal (base-10) hundreds
chart is a sub-chart of the hexadecimal ``hundreds'' (256es) chart:
\begin{center}\tablesize
\begin{tabular}{|c|c|c|c|c|c|c|c|c|c|c|c|c|c|c|c|}
\hline
 00 & 01 & 02 & 03 & 04 & 05 & 06 & 07 & 08 & 09 & 0a & 0b & 0c & 0d & 0e & 0f \\ \hline
 10 & 11 & 12 & 13 & 14 & 15 & 16 & 17 & 18 & 19 & 1a & 1b & 1c & 1d & 1e & 1f \\ \hline
 20 & 21 & 22 & 23 & 24 & 25 & 26 & 27 & 28 & 29 & 2a & 2b & 2c & 2d & 2e & 2f \\ \hline
 30 & 31 & 32 & 33 & 34 & 35 & 36 & 37 & 38 & 39 & 3a & 3b & 3c & 3d & 3e & 3f \\ \hline
 40 & 41 & 42 & 43 & 44 & 45 & 46 & 47 & 48 & 49 & 4a & 4b & 4c & 4d & 4e & 4f \\ \hline
 50 & 51 & 52 & 53 & 54 & 55 & 56 & 57 & 58 & 59 & 5a & 5b & 5c & 5d & 5e & 5f \\ \hline
 60 & 61 & 62 & 63 & 64 & 65 & 66 & 67 & 68 & 69 & 6a & 6b & 6c & 6d & 6e & 6f \\ \hline
 70 & 71 & 72 & 73 & 74 & 75 & 76 & 77 & 78 & 79 & 7a & 7b & 7c & 7d & 7e & 7f \\ \hline
 80 & 81 & 82 & 83 & 84 & 85 & 86 & 87 & 88 & 89 & 8a & 8b & 8c & 8d & 8e & 8f \\ \hline
 90 & 91 & 92 & 93 & 94 & 95 & 96 & 97 & 98 & 99 & 9a & 9b & 9c & 9d & 9e & 9f \\ \hline
 a0 & a1 & a2 & a3 & a4 & a5 & a6 & a7 & a8 & a9 & aa & ab & ac & ad & ae & af \\ \hline
 b0 & b1 & b2 & b3 & b4 & b5 & b6 & b7 & b8 & b9 & ba & bb & bc & bd & be & bf \\ \hline
 c0 & c1 & c2 & c3 & c4 & c5 & c6 & c7 & c8 & c9 & ca & cb & cc & cd & ce & cf \\ \hline
 d0 & d1 & d2 & d3 & d4 & d5 & d6 & d7 & d8 & d9 & da & db & dc & dd & de & df \\ \hline
 e0 & e1 & e2 & e3 & e4 & e5 & e6 & e7 & e8 & e9 & ea & eb & ec & ed & ee & ef \\ \hline
 f0 & f1 & f2 & f3 & f4 & f5 & f6 & f7 & f8 & f9 & fa & fb & fc & fd & fe & ff \\ \hline
\end{tabular}
\end{center}

One thing we enjoy at home is swapping numbers (in V's chart, the
numbers are on cards in pockets, so swaps are easy) and looking for
the anomalies.  Our friend Dustin Lang noted that certain swaps are
harder to find than others.  Here is one of his favorites:
\begin{center}\tablesize
\begin{tabular}{|c|c|c|c|c|c|c|c|c|c|}
\hline
 00 & 01 & 02 & 03 & 04 & 05 & 06 & 07 & 08 & 09 \\ \hline
 10 & 11 & 12 & 13 & 41 & 15 & 16 & 17 & 18 & 19 \\ \hline
 20 & 21 & 22 & 23 & 24 & 25 & 26 & 27 & 28 & 29 \\ \hline
 30 & 31 & 32 & 33 & 34 & 35 & 36 & 37 & 38 & 39 \\ \hline
 40 & 14 & 42 & 43 & 44 & 45 & 46 & 47 & 48 & 49 \\ \hline
 50 & 51 & 52 & 53 & 54 & 55 & 56 & 57 & 58 & 59 \\ \hline
 60 & 61 & 62 & 63 & 64 & 65 & 66 & 67 & 68 & 69 \\ \hline
 70 & 71 & 72 & 73 & 74 & 75 & 76 & 77 & 78 & 79 \\ \hline
 80 & 81 & 82 & 83 & 84 & 85 & 86 & 87 & 88 & 89 \\ \hline
 90 & 91 & 92 & 93 & 94 & 95 & 96 & 97 & 98 & 99 \\ \hline
\end{tabular}
\end{center}
My favorite is to swap ``68'' and ``89'' but turn the cards
upside-down as you do it.  On V's chart you can only find it by
noticing that the top circle on the ``8'' is slightly smaller than the
bottom circle!

\paragraph{Notes for the next draft of this document:}
Vectors of constant length and direction shifted around the chart
provide consistent differences [to demonstrate this graphically I need
  to make the hundreds charts in a plotting program].  This works for
1-indexed charts too.

Vectors coming from zero lead to multiplication tables, especially if
you think of the full two-dimensional plane as being tiled with charts
to the right, and in charts incremented by 100 going down.  This also
works for 1-indexed charts, as long as you make the vectors come from
the fictitious zero position.  That said, the zero-indexed chart is
nice here, because factors of 9 and 11 can be read off the chart's two
major diagonals.

These vector realizations work for any kind of chart, even ones that
aren't multiples of ten on each side, such as:
\begin{center}\tablesize
\begin{tabular}{|c|c|c|c|c|c|c|c|c|c|}
\hline
 00 & 01 & 02 & 03 & 04 & 05 & 06 \\ \hline
 07 & 08 & 09 & 10 & 11 & 12 & 13 \\ \hline
 14 & 15 & 16 & 17 & 18 & 19 & 20 \\ \hline
 21 & 22 & 23 & 24 & 25 & 26 & 27 \\ \hline
 28 & 29 & 30 & 31 & 32 & 33 & 34 \\ \hline
 35 & 36 & 37 & 38 & 39 & 40 & 41 \\ \hline
 42 & 43 & 44 & 45 & 46 & 47 & 48 \\ \hline
\end{tabular}
\end{center}
And just as the $10\times 10$ chart has factors of 10 down the left,
and factors of 9 and 11 down the two diagonals, this chart has factors
of 7 down the left and factors of 6 and 8 down the diagonals.

The chart is all about integers so far, but it can be generalized to
real numbers.  Just as the number 23 is plotted at $(x,y) = (3,2)$,
the number 23.743319 can be plotted at $(x,y) = (3.439, 2.731)$.
[Make a figure of this.]  The idea is to put the number in decimal
representation (or any other base $N$ for an $N\times N$ chart) and
then make the two numbers you get by taking every second digit from
the representation.  [Find out what this is called.]

\end{document}

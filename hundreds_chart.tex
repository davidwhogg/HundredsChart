% Copyright 2011 David W. Hogg (NYU).
% All rights reserved.

% to-do:
% ------
% - More subjects to discuss:
%   - You can go above 100.
%   - You can go into negative-number territory if you are brave!

\documentclass[12pt,pdftex]{article}
\usepackage{graphicx}
\newcommand{\tablesize}{\footnotesize\sffamily}
\newcommand{\showchart}[1]{\includegraphics[scale=0.67]{#1}}
\begin{document}

\subsection*{The hundreds chart}

\noindent
\textsl{by David W. Hogg, New York City, 2011 December 11}
\vspace{1ex}

In kindergarten, V wasn't showing a lot of interest in math, but she
was showing a lot of interest in ``playing school'' at home.  J had
the brilliant insight that all V needed was a hundreds chart like her
teacher Emily's.  It worked like a charm; V became much more
interested in math overnight.  She started spontaneously doing math
problems and started noticing patterns in the chart without any
prompting from us.  But \emph{I} also came to love the chart.  Here I
set down a few thoughts about it.

Here's the way the usual hundreds chart is arranged in most
classrooms:
\begin{center}
\showchart{hundreds_chart_standard.pdf}
\end{center}

Imagine instead that the chart started not in the top-left corner,
but one space to the right of that:
\begin{center}
\showchart{hundreds_chart_skipzero_nzp.pdf}
\end{center}
With this small change, the tens column (10, 20, 30, and so on) is on
the left instead of the right.  To my eye, it looks nicer, because,
for example, on the third row, every number starts with the digit
``2''.  Previously, they all began with ``2'' except the last number
(30), which begins with ``3''.

I love that every number on each row begins with the same digit, and
every number in each column ends with the same digit.  It is like the
second digit gives the ``$x$'' (horizontal) position, and the first
gives the ``$y$'' (vertical) position, as in a Cartesian coordinate
system.  This symmetry suggests typesetting the first row with
zero-prefaced numbers, like this:
\begin{center}
\showchart{hundreds_chart_skipzero.pdf}
\end{center}
This also suggests thinking about the ``transpose'' chart:
\begin{center}
\showchart{hundreds_chart_transpose_skipzero.pdf}
\end{center}
Nice!

Zero-prefacing the top row (or left column in the transpose chart)
makes sense from a symmetry sense, and also permits discovery of the
idea that a one-digit number has zero in the tens position implicitly.
It also suggests filling in the top-left blank with the only possible
number: It has to have a zero in the first position, and a zero in the
second position:
\begin{center}
\showchart{hundreds_chart_default.pdf}
\end{center}

But one thing I like about this hundreds chart is that on the
hundredth day of school, you don't complete this chart (that happened
on the 99th day); you \emph{start over}, by prepending ``1'' digits in
front of numbers.  This is what it could look like on the 100th day:
\begin{center}
\showchart{hundreds_chart_day100.pdf}
\end{center}
On the 114th day, it could look like this:
\begin{center}
\showchart{hundreds_chart_day114.pdf}
\end{center}

Going back to the basic hundreds chart, to wit:
\begin{center}
\showchart{hundreds_chart_default.pdf}
\end{center}
as a computer scientist (in part) I can't help notice that the
top-left $2\times 2$, $3\times 3$, and $4\times 4$ blocks of the
chart look like this:
\begin{center}
\showchart{hundreds_chart_2x2_base2.pdf}
\quad
\showchart{hundreds_chart_3x3_base3.pdf}
\quad
\showchart{hundreds_chart_4x4_base4.pdf}
\end{center}
These are the binary, trinary, and base-4 ``hundreds'' charts, or, if
you prefer, the binary fours chart, trinary nines chart, and base-4
sixteens chart.  And so on.  Also, the decimal (base-10) hundreds
chart is a sub-chart of the hexadecimal ``hundreds'' (256es) chart:
\begin{center}
\showchart{hundreds_chart_16x16_base16.pdf}
\end{center}

One thing we enjoy at home is swapping numbers (in V's chart, the
numbers are on cards in pockets, so swaps are easy) and looking for
the anomalies.  Our friend Dustin Lang noted that certain swaps are
harder to find than others.  Here is one of his favorites:
\begin{center}
\showchart{hundreds_chart_swap14.pdf}
\end{center}
My favorite is to swap ``68'' and ``89'' but turn the cards
upside-down as you do it.  On V's chart you can only find it by
noticing that the top circle on the ``8'' is slightly smaller than the
bottom circle!

The chart is mainly and usually used for teaching; in particular it
helps students get used to the numbers and their relationships; it is
really a ``wrapped'' number line.  Because of the regularity with
which the number line has been wrapped onto the chart, two-dimensional
vectors of constant length and direction shifted around the chart
provide consistent differences, like this:
\begin{center}
\showchart{hundreds_chart_tr_23.pdf}
\end{center}
Here I have shown two different kinds of vectors that correspond to
differences of 23, and I have shown the ``shadow'' chart that lies,
implicitly, to the right of the main chart.  That is, I have continued
the chart to the right artificially to make the relationship of the
two 23 vectors somehow more clear.  For any number difference there
are always two sensible arrows to draw, one going rightward and one
going leftward (and many more than two if you are willing to fully
tile the two-dimensional plane).

The fact that this tiling is useful---the fact that you go infinitely
far in the horizontal direction either way with hundreds
charts---shows that really the chart doesn't live in the plane; it
lives on a one-torus or cylinder: If you go far enough to the right or
left, you return to where you started, but one row down or up.  That
is, the chart really lives on a cylinder, and the cylinder has a
number line wrapped around it in a helix of slope $1/10$ (or $1/N$ for
base $N$).

But back to vectors: Any triangle on the chart represents a triplet of
integers that sum to zero.
\begin{center}
\showchart{hundreds_chart_triangle.pdf}
\end{center}
This triangle (above) represents the relationship $23-29+6=0$.  More
generally, any $n$-sided polygon represents a set of $n$ integers that
sum to zero.  This kind of arithmetic has the strange property that
(for some, anyway), vector addition can be used to help with integer
addition; most conventional thought would put these mathematical
subjects in the opposite order!

In this pedagaogical spirit, really---and I should have put this
ealier---the chart perhaps should count up from the bottom, so ``up''
is ``larger'' like with floors in a building.
\begin{center}
\showchart{hundreds_chart_bottomup.pdf}
\end{center}
J points out that this is particularly important when the chart is
used to teach or help support subtraction: You often find yourself
saying to a child ``now go down by ten'' for ``subtract ten''; that's
pretty confusing when ``down'' is ``up''.  Going against this thought
is the point that the hundreds chart lives in a classroom where the
students are \emph{also} learning their reading and writing skills; we
read (English) \emph{down} the page.

Vectors coming from zero lead to multiplication tables, especially if
(once again) you think of the full two-dimensional plane as being
tiled with charts to the right.  Here are is the 13 times table:
\begin{center}
\showchart{hundreds_chart_tr_13s.pdf}
\end{center}
The easiest way to see these
multiplications is to have a zero-indexed chart (my preference),
however this all also works for 1-indexed charts, as long as you make
the vectors come from the fictitious zero position, which lies just
above the 10.
\begin{center}
\showchart{hundreds_chart_tr_index1_13s.pdf}
\end{center}

Another respect in which the symmetry of the zero-indexed chart is
nicer---related to multiplication---is that factors of 9 and 11 can be
read off the chart's two major diagonals.  The diagonals are
interesting on any any kind of zero-indexed chart, even ones that
aren't $10\times 10$ (or $N\times N$ for base $N$):
\begin{center}
\showchart{hundreds_chart_7x7_base10.pdf}
\end{center}
Just as the $10\times 10$ chart has factors of 10 down the left, and
factors of 9 and 11 down the two diagonals, this $7\times 7$ chart
(above) has factors of 7 down the left and factors of 6 and 8 down the
diagonals.

The chart has been all about integers so far, but it can be
generalized to real numbers.  Just as the number 23 is plotted at
$(x,y) = (3,2)$, the number 23.743319 can be plotted at $(x,y) =
(3.439, 2.731)$, where we made the $x$ coordinate by taking the ones,
hundredths, ten-thousandths, and millionths places and collapsing them
into a new decimal number, and we made the $y$ coordinate by taking
the tens, tenths, thousandths, and hundred-thousandths places.  Here
is that location (marked ``$\mathbf Q$'') plotted on top of a standard chart:
\begin{center}
\showchart{hundreds_chart_Q.pdf}
\end{center}
In general the idea is to write the number in the current
representation (base $10$ in this case) and then make the two numbers
you get by taking every second digit from the representation.  This is
the simplest example of a mapping from the real number line (a
one-dimensional object) to the plane (a two-dimensional object).  This
mapping has the beautiful property that all of the vector addition
properties mentioned above work for real numbers just as well as they
do for the integers.

There are many such mappings; the standard chart is just one example of
an enormous class of mappings from one dimension to two.  For example,
the chart can be rearranged to represent a fractal-like space-filling
curve.  For the simplest space-filling curve I know, it makes sense to
do a $2^n\times 2^n$ chart, like this $8\times 8$ chart (still in
base-10 numbers though):
\begin{center}
\showchart{hundreds_chart_8x8_base10_sfc.pdf}
\end{center}
Here is a bigger space-filling curve chart in hexadecimal:
\begin{center}
\showchart{hundreds_chart_16x16_base16_sfc.pdf}
\end{center}
These space-filling curve charts have very interesting properties;
they don't have the vector-space properties we used above to show
addition and multiplication, but they have the local property that
nearby numbers are guaranteed to be nearby, which is not true in the
standard chart (think the separation of 39 and 41 on the standard
$10\times 10$ chart).  Space-filling makes the chart (probably)
useless for primary education, but way cool!

\end{document}

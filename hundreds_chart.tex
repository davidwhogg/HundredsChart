% Copyright 2011 David W. Hogg (NYU).
% All rights reserved.

% to-do:
% ------
% - expand last section into real paragraphs

\documentclass[12pt,pdftex]{article}
\usepackage{graphicx}
\newcommand{\tablesize}{\footnotesize\sffamily}
\newcommand{\showchart}[1]{\includegraphics[scale=0.67]{#1}}
\begin{document}

\subsection*{The hundreds chart}

\noindent
\textsl{by David W. Hogg, New York City, 2011 July 30}
\vspace{1ex}

In kindergarten, V wasn't showing a lot of interest in math, but she
was showing a lot of interest in ``playing school'' at home.  J had
the brilliant insight that all V needed was a hundreds chart like her
teacher Emily's.  It worked like a charm; V became much more
interested in math overnight and does math problems and notices
patterns in the chart voluntarily and without any prompting from us.
But \emph{I} also came to love the chart.  Here I set down a few
little thoughts about it.

Here's the way the usual hundreds chart is arranged in most
classrooms:
\begin{center}
\showchart{hundreds_chart_standard.pdf}
\end{center}

Imagine instead that the chart started not in the top-left corner,
but one space to the right of that:
\begin{center}
\showchart{hundreds_chart_skipzero_nzp.pdf}
\end{center}
With this small change, the tens column (10, 20, 30, and so on) is on
the left instead of the right.  To my eye, it looks nicer, because,
for example, on the third row, every number starts with the digit
``2''.  Previously, they all began with ``2'' except the last number
(30), which begins with ``3''.

I love that every number on each row begins with the same digit, and
every number in each column ends with the same digit.  It is like the
second digit gives the ``$x$'' (horizontal) position, and the first
gives the ``$y$'' (vertical) position, as in a Cartesian coordinate
system.  This symmetry suggests typesetting the first row with
zero-prefaced numbers, like this:
\begin{center}
\showchart{hundreds_chart_skipzero.pdf}
\end{center}
This also suggests thinking about the ``transpose'' chart:
\begin{center}
\showchart{hundreds_chart_transpose_skipzero.pdf}
\end{center}
Nice!

Zero-prefacing the top row (or left column in the transpose chart)
makes sense from a symmetry sense, and also permits discovery of the
idea that a one-digit number has zero in the tens position implicitly.
It also suggests filling in the top-left blank with the only possible
number: It has to have a zero in the first position, and a zero in the
second position:
\begin{center}
\showchart{hundreds_chart_default.pdf}
\end{center}

But one thing I like about this hundreds chart is that on the
hundredth day of school, you don't complete this chart (that happened
on the 99th day); you \emph{start over}, by prepending ``1'' digits in
front of numbers.  This is what it could look like on the 100th day:
\begin{center}
\showchart{hundreds_chart_day100.pdf}
\end{center}
On the 114th day, it could look like this:
\begin{center}
\showchart{hundreds_chart_day114.pdf}
\end{center}

Going back to the basic hundreds chart, to wit:
\begin{center}
\showchart{hundreds_chart_default.pdf}
\end{center}
as a computer scientist (in part) I can't help notice that the
top-left $2\times 2$, $3\times 3$, and $4\times 4$ blocks of the
chart look like this:
\begin{center}
\showchart{hundreds_chart_2x2_base2.pdf}
\quad
\showchart{hundreds_chart_3x3_base3.pdf}
\quad
\showchart{hundreds_chart_4x4_base4.pdf}
\end{center}
These are the binary, trinary, and base-4 ``hundreds'' charts, or, if
you prefer, the binary fours chart, trinary nines chart, and base-4
sixteens chart.  And so on.  Also, the decimal (base-10) hundreds
chart is a sub-chart of the hexadecimal ``hundreds'' (256es) chart:
\begin{center}
\showchart{hundreds_chart_16x16_base16.pdf}
\end{center}

One thing we enjoy at home is swapping numbers (in V's chart, the
numbers are on cards in pockets, so swaps are easy) and looking for
the anomalies.  Our friend Dustin Lang noted that certain swaps are
harder to find than others.  Here is one of his favorites:
\begin{center}
\showchart{hundreds_chart_swap14.pdf}
\end{center}
My favorite is to swap ``68'' and ``89'' but turn the cards
upside-down as you do it.  On V's chart you can only find it by
noticing that the top circle on the ``8'' is slightly smaller than the
bottom circle!

\paragraph{Notes for the next draft of this document:}
Vectors of constant length and direction shifted around the chart
provide consistent differences, like this:
\begin{center}
\showchart{hundreds_chart_tr_23.pdf}
\end{center}
This works for 1-indexed charts too.  Along these lines, any triangle
on the chart represents a triplet of integers that sum to zero.  Any
$n$-sided polygon represents a set of $n$ integers that sum to zero.

Vectors coming from zero lead to multiplication tables, especially if
you think of the full two-dimensional plane as being tiled with charts
to the right, and in charts incremented by 100 going down.  This also
works for 1-indexed charts, as long as (once again) you make the
vectors come from the fictitious zero position.  That said, the
zero-indexed chart is nice here, because factors of 9 and 11 can be
read off the chart's two major diagonals.

These vector realizations work for any kind of chart, even ones that
aren't multiples of ten on each side, such as:
\begin{center}
\showchart{hundreds_chart_7x7_base10.pdf}
\end{center}
And just as the $10\times 10$ chart has factors of 10 down the left,
and factors of 9 and 11 down the two diagonals, this chart has factors
of 7 down the left and factors of 6 and 8 down the diagonals.

The tiling of two-dimensional space with hundreds charts---which we
used to demonstrate vector properties---really shows that the chart
lives on a one-torus or cylinder, with the number line wrapped around
it in a helix of slope $1/10$ (or $1/N$ for base $N$).

You can go into negative-number territory if you are brave.

The chart is all about integers so far, but it can be generalized to
real numbers.  Just as the number 23 is plotted at $(x,y) = (3,2)$,
the number 23.743319 can be plotted at $(x,y) = (3.439, 2.731)$.
[Make a figure of this.]  The idea is to put the number in decimal
representation (or any other base $N$ for an $N\times N$ chart) and
then make the two numbers you get by taking every second digit from
the representation.  [Find out what this is called.]

Generalizing this further, the chart can be rearranged into a
fractal-like space-filling curve, as in my blog post on hoggmaker.
This leads to various interesting properties for any chart that is
$2^n\times 2^n$ for some integer $n$.  You lose the vector-space
properties, but you gain locality.  That is, it makes the chart
useless for primary education, but way cool!

Really---and I should put this ealier---the chart should count up from
the bottom, so ``up'' is ``larger'' like with floors in a building.  J
points out that this is particularly important for subtraction, when
you say to a child ``now go down by ten'' for ``subtract ten''.

\end{document}
